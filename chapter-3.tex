\chapter{时代社会}

\section{规则}

\section{民众、发声}
\subsection{句子}
\begin{enumerate}
\item 胸怀梦想者,逐梦前行;登高瞭望者,声震长空。
\item 你在荒村里忍着饥肠,你常常想到死填沟壑。你却不断地唱着哀歌,为了人间壮美的存亡。\by{冯志《十四行集》,纪念杜甫}
\item 创造新陆地的,不是那滚滚的波浪,却是它底下细小的细沙。\by{《繁星·春水》}
\item 评论家经常不看作家的作品,却声称自己看了;作家经常看评论家的评论,却声称从来不看。\by{王晓渔}
\end{enumerate}
\subsection{人物例子}
\rw{纳迪亚·穆拉德}{作为一位从ISIS性奴到诺贝尔和平奖获得者的女性,她一直在为和自己一样的女性发声。“讲述自己的故事从来都不是一件容易的事。每讲一次,你都要重新再体会一遍。”}
\rw{樋口一叶}{她的头像被印在日元纸币上。在她的时代,男权至上,于是她用自己的笔,为女性呐喊。虽然她没有挑战当时社会的力量,但也给女性带来一种精神上的安慰。}


\section{教育}
\subsection{句子}
\begin{enumerate}
\item 国将兴,必贵师而重傅。贵师而重傅,则法度存。\by{荀子}
\item 有人曾形象地说,这就好像是看表演。第一排的人站起来了,后面的人不得不也站起来。结果大家都站着看完了,累得要命。\by{评“吃苦教育”}
\item 我们现在缺少的不是应试教育,我们缺少的是科学教育。\by{周有光}
\item 安全时锻炼孩子,危险时保护孩子,这是狼的教子之道。\by{托尔斯泰}
\end{enumerate}
\subsection{人物、事例}
\rw{叶连平}{他从学校退休后,坚持为学生义务补课。2000年,他在家里办了“留守儿童之家”,从未落下一堂课,却从未收过一分钱。}
\rw{“六岁娃喝两瓶啤酒”事件}{孩子未来的路需要孩子去走,不能任由家长去自私、残忍地剥夺,不能让童心被趋利的家长污染。}
\rw{教师李某打人事件}{
\begin{itemize}
\item 家长:“与其让孩子受明天的苦,不如让孩子受今天的苦”的无奈。
\item 教育:体制存在问题、成绩决定论、教育方法等。
\item 不能开了(教师过度惩罚学生却仍守教职)的先例。
\item 家长们也是有苦衷的……媒体也不应该批评家长。
\end{itemize}}

\section{农业、科学技术}
\subsection{句子}
\begin{enumerate}
\item 农者,天下之本也。\by{欧阳修}
\item 农业和农村就像大树的根,这个跟扎牢了,大家的日子才会过得更好,国家也才能发展的更好。\by{李克强}
\item 互联网其实是一种方法论、价值观,就是你怎么用互联网的思维去考虑问题。\by{雷军}
\end{enumerate}
\subsection{例子}
\rw{卢旺达Rwesero}{这里种植了17 000多棵咖啡树,因为交通不便,以前将咖啡卖到首都都难,现在通过天猫国际,咖啡种植户可以将咖啡卖到遥远的中国。“我的孩子们都能上学了……这改变了我的生活。”}
\rw{法拉第}{他是应该被铭记的电气化功臣。他出生于英国一个铁匠家庭,曾经在装订工厂当学徒。他利用这个条件,读了很多科学书籍,从中获得了丰富知识。}





