\chapter{个人处世}
时代像筛子,筛得每个人流离失所,筛得少数人出类拔萃。\by{王鼎钧}
\section{“时代青年”例子聚合}
一个时代的精神,是青年代表的精神;一个时代的性格,是青年代表的性格。\by{马克思}
\rw{曹原}{14岁考入中国科学技术大学“少年班”,后前往美国麻省理工大学攻读博士学位,发现石墨烯的“魔法角度”,连续在《科学》上发表两篇论文,大大推进了高温超导材料的研发。}
\rw{万蕊雪}{清华大学博士,26岁时在《科学》以第一作者身份发表6篇文章,揭示了生命大分子剪接体结构等科学问题。2016年入选中国科协“未来女科学家计划”,2018年青年科学家奖胜出者。}
\rw{刘明侦}{2012年在牛津大学光伏光电研发中心,以新型太阳能电池为研究核心取得一系列创新性研究成果。2013年9月,其首篇基于平板薄膜结构的卤化物钙钛矿太阳能电池文章在《自然》主刊上发表。}
\rw{舒畅}{舒畅及其团队开始技术攻关,进行自主研发。2017年,零壹空间OS-X火箭“重庆两江之星”在中国西北某基地成功点火升空,成功实现了中国首枚民营自研商用亚轨道火箭首飞,开创了中国商业航天历史上的发展奇迹和崭新起点。}
\rw{宋玺}{2013年就读于北京大学,2015年参加海军。曾是北京大学学生合唱团领唱,亚丁湾护航唯一女陆战队员。}
\rw{申怡飞}{中国5G核心技术最年轻的开拓者,17岁师从“中国4G技术掌门人”尤肖虎教授。在2019年年初,申怡飞团队开创的极化码方案被写入5G行业标准。}
\rw{沈亦晨}{2014年,沈亦晨在《科学》杂志上以第一作者身份发表了题为《宽波段光学的角度选择》的论文,首次实现了在材料尺度上对光的传播方向的控制。读博期间,沈亦晨共发表了25篇顶级期刊论文,提交了10项美国专利。}



\section{奋斗、毅力、志向、忧患意识}
\subsection{句子}
\begin{enumerate}
\item 我从来不相信什么懒洋洋的自由。我向往的自由是通过勤奋和努力实现的更广阔的人生。\by{设计师{  }山本耀司}
\item 伟大的事业之所以伟大,不仅因为这种事业是正义的、宏大的,而且因为这种事业不是一帆风顺的。伟大的人物之所以伟大,不仅因为这样的人物为人民、为人类建立了丰功伟绩,而且因为这样的人物在艰苦磨砺中铸就了坚强意志和高尚人格。\by{习近平}
\item 高山仰止,景行行止;虽不能至,心向往之。\by{司马迁}
\item 每一个不曾起舞的日子,都是对生命的辜负。\by{尼采}
\item 没有最终的成功,也没有最致命的失败,最可贵的是继续前进下去的勇气。\by{丘吉尔}
\item 我们从忧患中学得智慧,苦痛中炼出美德。\by{杨绛}
\item 斗争的生活使你干练,苦闷的煎熬使你醇化。这是时代要造就青年为能担负历史使命的两件法宝。\by{矛盾}
\item 如果你的脑海有个声音说:“你根本就画不成”。那无论如何,你都要去画,然后你就会发现那个声音消失了。\by{凡·高}
\end{enumerate}

\subsection{人物}
\rw{苏武}{十九年如一日,他牧羊于大漠,孤独与寂寞伴他左右,却从没想过屈服。时间没有磨灭他的气节,孤独没有打败他的信念。}
\rw{阿列克谢·梅列西耶大(苏联飞行员)}{苏联卫国战争期间,他的飞机被击落,他在左腿受伤、冻坏的情况下爬行18个昼夜,返回了自己的阵地。双腿截肢后,他经过训练,又加入驾驶战斗机作战。}
\rw{欧·亨利}{他不想当父亲安排的小伙计,他想当大作家。他把在监狱听说的故事写成精炼的短篇小说,托监狱外面的朋友拿去发表。后来,他的名字引起了文坛的注意,最终,他由监狱走向了文坛。}
\rw{《史记·报任安书》}{盖文王拘而演《周易》,仲尼厄而作《春秋》,屈原放逐,乃赋《离骚》,左丘失明,厥有《国语》,孙子膑脚,《兵法》修列,不韦迁蜀,世传《吕览》,韩非囚秦,《说难》《孤愤》;《诗》三百篇,大抵圣贤发奋之所为作也。}
\rw{曾孝濂}{“中国植物画第一人”,《中国植物志》插图画师。“既要坐得冷板凳,也要登得大山头。”}


\section{奉献、自我牺牲、宽容待人、考虑他人}
生活如山,有人岁月静好,有人负重前行。
\subsection{句子}
\begin{enumerate}
\item 为众人抱薪者,不可使其冻毙于风雪。
\item 执心坚白,謇謇非躬。\\
(志节坚贞,不可动摇,为国为民,不为自身。)
\item 苟利国家生死以,岂因祸福避趋之。\by{林则徐}
\item 人家是说了再做,我是做了再说。人家说了也不一定做,我是做了也不一定说。\hfill ——闻一多
\item 不要习惯了黑暗就为黑暗辩护,不要为了自己的苟且而得意,不要嘲笑那些比自己更勇敢热情的人们。\hfill ——曼德拉
\item 伟大的事业之所以伟大,不仅因为这种事业是正义的、宏大的,而且因为这种事业不是一帆风顺的。伟大的人物之所以伟大,不仅因为这样的人物为人民、为人类建立了丰功伟绩,而且因为这样的人物在艰苦磨砺中铸就了坚强意志和高尚人格。\by{习近平}
\item (医生)这些人脱下白大褂后,其实与你我毫无二致。(他们)会累、会哭、会颓丧。但穿上白大褂,他们是英勇的武士,背负着使命和责任,坚定地奔赴着属于他们的战场。
\item 和平年代,负重前行者数不胜数,就像英雄远不止出现在战争年代。
\item 无穷的远方,无数的人们,都与我有关。\by{鲁迅}
\item 虽千万人,吾往矣。
\item 世界上有这样一些幸福的人,他们把自己的痛苦化作他人的幸福,他们挥泪埋葬了自己在尘世间的希望,却变成了种子,长出鲜花和香膏,为孤苦伶仃的人医治创伤。\by{《汤姆叔叔的小屋》}
\end{enumerate}

\subsection{人物}
\rw{叶连平}{他从学校退休后,坚持为学生义务补课。2000年,他在家里办了“留守儿童之家”,从未落下一堂课,却从未收过一分钱。}
\rw{付小洁}{她是“世界上海拔最高的火车站”——青藏铁路唐古拉车站的厨师,每天忍受着剧烈的高原反应为大家做饭。清晨,她就做好午餐装进保温盒,临行时一一叮嘱:捂严实,别洒了。}
\rw{钟南山}{2003年非典,他在自己的岗位做到了最好,也因此获得了极高的声誉。而2020年的新冠肺炎之下,他又到了抗疫的最前线。尽管年岁已高,却仍然为抗疫奉献自我。}
\rw{程科意}{1月25日,曾经当过空降兵的滴滴司机程科意,自告奋勇地上了“前线”,加入了武汉社区保障队。他说:“(我)应该要积极地站出来,把我们共同的家维护好。”}
\rw{邓世昌}{在邓世昌20多年的海军服役生涯里,邓世昌仅回乡探亲3次,最长的一次,也只不过在家里住了7天。中法战争时期,邓世昌的父亲去世,但面对严峻的海防形势,他强忍悲痛没有回家。}
\rw{于谦}{在任上,他平反冤狱、赈济灾荒、治理水患、发展生产,深受爱戴。}
\rw{叶沙}{叶沙是一个热爱篮球的少年。16岁的他突发脑溢血离世,而他捐献出的心脏、肝脏、肺脏、肾脏和眼角膜,却让7个人重获新生。其中的5名受捐者,组成了“叶沙队”,走进了WCBA全明星赛场。}
\rw{窦权利}{他是西安市胸科医院结核科主任。多年来,他一直坚持手绘简图向患者解释病情。三十年来,他画了上万个肺。“这样的窦主任,是上天派来拯救结核病人的救星。”}
\rw{刘增盛}{老人每天都乘坐公共交通工具,往返于市区与旅顺之间。他挂着“勿需让座”的牌子乘地铁,被称为“硬核大爷”。}
\rw{一位外卖顾客}{重庆一外卖小哥因车子坏了,送外卖迟到了40分钟。女顾客不但没有责备,反而还多给了他6元钱小费。}

\section{隐忍、成熟、认识自我、认识方向}
\subsection{句子}
\begin{enumerate}
\item 镜子里的人,是显而易见的,自己却看不真。\by{杨绛}
\item 成熟是一种明亮而不刺眼的光辉,一种圆润而不腻耳的声响,一种不再需要对别人察言观色的从容,一种终于停止向周围申诉求苦的大气,一种不理会喧闹的微笑,一种洗刷了偏激的淡薄,一种无需声张的厚实,一种能看得很远却并不陡峭的高度。\by{余秋雨}
\item 使人成熟的是经历,而不是岁月。
\item 人一生中有两个重要的日子,一个是你出生的日子,另一个是你意识到自己为何出生的日子。\by{马克·吐温}
\item 谁终将声震人间,必长久深自缄默;谁终将点燃闪电,必长久如云漂泊。\by{尼采}
\item 大部分人在二三十岁的时候就死了。因为过了这个年龄,他们只是自己的影子,此后的余生则是在模仿自己中度过。\by{罗曼·罗兰}
\item 一粒尘埃,在空气中凝结,最后生成磅礴的云雨;一粒沙石,在蚌体内打磨,最后结成昂贵的珍珠。有时候,渺小的开始可以成就雄壮而宏大的事业;有时候,平凡的开始可以走出崇高而伟大的人生。
\end{enumerate}
\subsection{例子}
\rw{伯格曼}{他是天才型的导演。在很小的时候,他就明确地发现了自己的兴趣所在,并且为之付出了一生的努力。凭借对电影的热爱,以及对自己最苛刻的要求,他在日复一日中成就了伯格曼传奇。}
\rw{章鱼}{一只章鱼的体重可以达到70磅。但是,如此庞大的家伙,身体却非常柔软,柔软到几乎可以钻进任何想去的地方。它们最喜欢做的事情,就是将自己的身体钻进海螺壳里躲起来。等到鱼虾游进,便能每餐一顿。与此同时,渔民也掌握了它们的天性,他们用小绳子将小瓶子串在一起沉入海底,章鱼一看见小瓶子便往里钻,最终成了瓶子里的囚徒、人类的美餐。}


\section{创新、熟能生巧、工匠精神、敬业}
\subsection{句子}
\begin{enumerate}
\item 周虽旧邦,其命维新。\by{《诗经·大雅·文王》}
\item 创新是一个国家进步的灵魂,是国家兴旺发达的不竭动力。如果自主创新能力上不去,一味靠技术引进,就永远难以摆脱技术落后的局面。一个没有创新能力的民族,难以屹立于世界先进民族之林。\by{江泽民}
\item 我们说相声,学的是技术,练的是手艺啊!这和炸油条一样,一个炸油条的会担心自己江郎才尽吗?把相声当手艺,不当才气。才气会尽,手艺只会越来越精尽。\by{郭德纲}
\item 一个好的工匠,其实就是一个艺术家。\by{清华大学美术学院{  }杨阳}
\item 从60\%到90\%固然优秀,但是从99\%到99.99\%的态度才是真正的工匠精神。工匠的核心不是去“制造”什么,而是一种追求卓越的心态。\by{李光斗}
\item 那些作画时单凭实践和肉眼的判断,而缺乏理性的画家,只会抄袭而一无所知。\by{达·芬奇}
\item 领袖和跟风者的区别就在于创新。\by{乔布斯}
\end{enumerate}
\subsection{例子}
\rw{达·芬奇}{因为对作品的精益求精,达·芬奇是“重度的拖延症患者”。}
\rw{窦立国}{他手绘的快递送货“拥堵地图”,成为“同事公认的宝贝,很实用”,被公司作为培训新人的“教材”之一。}
\rw{蒯翔}{永乐年间的故宫设计师。他在建筑学上的造诣达到了炉火纯青的地步。施工前精确计算、竣工后分毫不差,巧妙使用榫卯结构、巧妙运用江南的建筑艺术,在历史的机遇面前,他创造了一个建筑史上的奇迹。}
\rw{马钧}{古代机械师,改进了织绫机,制造了指南车,改进了诸葛连弩和投石机,“天下皆服其巧也”。}
\rw{乔素凯}{他与“核”共舞26年,连续56 000步操作零失误。}
\rw{一个犹太人}{他将美国政府为了翻新自由女神像产生的大量废料铸成小自由女神像等工艺品出售,让一堆废料在三个月内变成了350万美金,每磅铜的价格翻了一万倍。}
\rw{胡玮炜}{她是摩拜单车创始人。摩拜已经成为了全球最大的智能共享单车运营平台,也成为全球最大的移动物联网运营平台之一。}
\rw{日本:18年获得18个诺贝尔奖}{日本每年都会发布《科学技术白皮书》,目的是从与其他国家的比较中看到自己的不足。}


\section{母爱、亲情}
\subsection{句子}
\begin{enumerate}
\item 所谓母亲是没有追求的。我的孩子将如何伟大,会如何富有,都不重要。心底深深希望的是,他每天都能健康、快乐,就算是再昂贵的礼物,也不及我的孩子可以心地善良、幸福美满地度过一生。\by{中川雅也《东京塔》}
\item 在你出生之前,你的父母并不像现在这样乏味,他们变成现在这个样子,是因为这些年来一直在为你付账单,给你洗衣服。所以在对父母喋喋不休之前,还是先去打扫一下你自己的屋子吧。\by{比尔·盖茨}

\end{enumerate}

\section{诚信、学术不端(造假)}
\subsection{句子}
\subsection{例子}
\begin{itemize}
\item 2019年2月,博士生演员翟天临被曝光学术论文抄袭造假,其从北大光华管理学院退站,北影撤销其学位。
\item 2018年10月,南京大学教授梁莹被曝光多篇论文涉嫌抄袭,一稿多投。
\item 2019年3月,湖南大学刘梦洁被指责硕士论文剽窃,毕业论文被湖大从知网撤下。
\item 2014年10月,日本细胞生物学研究员小保方晴子的论文被认定有篡改、捏造等问题,博士学位被早稻田大学撤销。一直为她的实验出力的导师自杀。
\item 2002年,美国的舍恩通过伪造数据,用所谓的“分子晶体管”糊弄包括权威期刊编辑在内的许多人,甚至在不同的论文中使用一样的数据,最终东窗事发,震惊整个科学界。此事成为“物理学界50年来最大的造假案件”。
\item “全球论文的撤稿量已经从2000年的每年不足40篇,上升至2018年的每年1 400篇。”
\end{itemize}

\section{豁达、乐观、取舍}
\subsection{句子}
\begin{enumerate}
\item 别为打翻的牛奶哭泣。\by{英国谚语}
\item 人生虽不快乐,而仍能乐观。\by{钱锺书}
\item 如果命运是世界上最烂的编剧,你就要争取做你自己人生最好的演员。\by{撒贝宁}
\item 就算是最有成就的科学家,得以实现的建议、猜想、愿望和初步判断,也不到十分之一。\by{法拉第}
\item 有所取必有所舍,有所紧必有所宽。\by{苏轼}
\end{enumerate}
\subsection{例子}
\rw{杨绛}{她“文革”的时候被安排打扫厕所,却把原本脏乱不堪的厕所打扫的干干净净。她说,“我自从做了‘扫厕所的’便乐的放肆”。}
\subsubsection{“寒门贵子”聚合}
\begin{itemize}
\item 庞众望\hspace{3ex}母亲卧病在床,父亲精神分裂,生活仅仅靠爷爷奶奶维持。沧州市理科状元。
\item 王心仪\hspace{3ex}一家六口全靠家中5亩地和父亲外出打零工维持生计。707分被北大中文系录取。
\item 何江\hspace{3ex}父母以种田打渔为生。从中国科学技术大学毕业后,赴哈佛大学研读生物化学,硕博连读,后在MIT从事博士后研究。
\end{itemize}
\rw{周有光}{他是“汉语拼音之父”,年轻时身体很弱,算命先生说他只能活到35岁。可他却活了111岁,在2017年离世。“不能怪算命先生不准,主要是因为科学进步、医学进步,所以我才能长寿。还有大概就是上帝把我给忘记了。}

\section{艺术、美感、仪式感}
\subsection{句子}
\begin{enumerate}
\item 艺术是另一世界里的东西,对于实际人生没有引诱性。\by{朱光潜}
\item 要求人生净化,先要求人生美化。
\item 无论是讲学问还是做事业的人都要抱有一副“无所为而为”的精神,把自己所做的学问事业当成一件艺术品看待。只求满足理想和情趣,不斤斤于利害得失,才可以有一番真正的成就。\by{朱光潜}
\item 我们日常忙碌生活的自我并不是完全真正的自我。在生活的追求中我们已经丧失了一些东西。\by{林语堂《生活的艺术》}
\item 能脱俗便是奇,作意尚曲折,不为奇而异;不合污便是清,绝俗求清者,不为清为激。\by{《菜根谭》}
\item 我们需要仪式……它使得某个日子区别于其他日子,某个时刻不同于其他时刻。\by{狐狸(出自《小王子》)}
\end{enumerate}
\subsection{事例}
\rw{陈波(警察)}{2018年10月23日晚,面对拿着捡到的5角钱准备交给警察的两个小朋友,他和他的同事们决定给他们鼓励,拿出接警本,认真地询问了小朋友的姓名、住址等。他说,这种做好事的仪式感能鼓励他们拾金不昧的精神。}
