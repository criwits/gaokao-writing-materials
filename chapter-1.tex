\chapter{家国文化}

\section{2019年度汉字:爱、稳、融、创、难}
\begin{enumerate}
\item[\textbf{爱}] 寒风中送餐的外卖小哥,大山中奋不顾身的消防队员,地铁里用臂膀支持病弱老父的青年,国旗下满腔赤诚的14亿护旗手……爱的内涵在2019年迭代升华。
\item[\textbf{稳}] 2019年世界多变,波澜起伏。漩涡中的中国从容应对:稳就业、稳金融、稳外贸、稳预期,任尔东西南北风,我自岿然不动。处变不惊、稳中求进才能更好地迎接挑战。
\item[\textbf{融}] 中外融合、军民融合、城乡融合、产业融合、媒体融合……“融”是中国高质量发展的创新之路;经济融合、文化融合、我中有你、你中有我……“融”是世界和平共生的旗帜。
\item[\textbf{创}] 5G、区块链、量子技术、人工智能……从无到有,中国在第四次工业革命中奋勇争先。强调基础研究和原始创新,更要产业创新和科技创新,自我升级,唯创新能激发生机。
\item[\textbf{难}] 新冠肺炎、贸易摩擦、全球防务关系复杂……全球的双边、多边关系与地区形势难上加难。
\end{enumerate}

\section{疫情好句}
\begin{enumerate}
\item 最初,没有人在意这一场灾难,这不过是一场山火、一次旱灾、一个物种的灭绝、一座城市的消失,直到这场灾难和每个人息息相关。\by{《流浪地球》}
\item 都说时代匆匆,但时代哪有脚,走的总是人。
\item 我们不仅需要用强健体魄支撑美好生活,而且需要把健康底色涂抹在历史车轮上,一起滚滚向前。
\item 你没有落后,你没有领先。在命运为你安排的属于自己的时区里,一切都准时。在自己的时区里寻觅时光的意义,正是彰显人生理念的价值标尺。
\item 你所站立的地方,正是你的中国;你怎么样,中国便怎么样;你是什么,中国便是什么;你有光明,中国便不黑暗。\\
(以上四句,出自《人民日报写给青年的八封信》)
\item 山川异域,风月同天。岂曰无衣,与子同裳(袍)。
\item 愿我们在硝烟尽散的世界里重逢。\by{木苏里}
\item 无穷的远方,无数的人们,都与我有关。\by{鲁迅}
\item 虽千万人,吾往矣。
\item 不要过分陶醉于我们对自然界的胜利,对于每一次这样的胜利,自然界都报复了我们。\by{恩格斯}
\item 离开是为了更好地相聚。逆行,在这个冬天格外壮美又令人动容。
\end{enumerate}

\section{爱国精神}

\subsection{句子}
\begin{enumerate}
\item “爱国”一词,源出《战国策》中“今秦虎狼之国也,兼有吞周之意……周君岂能无爱国哉?”本指秦国势强、野心勃勃,周天子怎能不吝惜国土,有所防备?
\item 每个人都有自己的故乡。当你转身离开故乡时,故乡就在你身后。当你走得再远点,身后就是你的国家。当你回头看到故乡、祖国,再转头向前看的时候,你会想到强大的祖国在你身后支持你。这种幸福感,是每一个人都心潮澎湃的原因。\hfill ——导演{ }卫铁
\item 我们爱我们的民族,这是我们自信心的源泉。\by{周恩来}
\item 一个人只要钟爱自我的祖国,有一颗爱国之心,就什么事情都能解决。什么苦楚、什么冤屈都受得了。\by{冰心}
\item 爱国主义就是千百年来巩固起来的对自己的祖国的一种深厚的感情。\by{列宁}
\item 爱祖国高于一切。\by{肖邦}
\item 国人无爱国心者,其国恒亡。\by{李大钊}
\item
时穷节乃见,一一垂丹青。
在齐太史简\textit{(崔杼弑君)},在晋董狐笔\textit{(书法不隐,良史也)}。
在秦张良椎\textit{(刺秦)},在汉苏武杰。
为严将军头\textit{(无降将军,断头将军)},为嵇侍中血\textit{(嵇绍以身卫皇帝)}。
为张雎阳齿\textit{(每战大呼,齿牙皆碎)},为颜常山舌\textit{(颜杲卿被俘虏,骂不绝,贼勾断其舌,含胡而绝)}。
或为辽东帽\textit{(管宁)},清操历冰雪。
或为出师表,鬼神泣壮烈。
或为渡江楫,慷慨吞胡羯\textit{(祖逖)}。
或为击贼笏,逆竖头破裂\textit{(段秀实)}。
是气所磅礴,凛冽万古存。\hfill ——文天祥《正气歌》
\item 爱国主义是我们民族精神的核心,是中华民族团结奋斗、自强不息的精神纽带。\by{习近平}
\item 我中国居于亚洲之东,气候温和,土地广博,人民繁多。五千年前,文化已开,地球上最有名之古国也。自我远祖以来,居于是,衣于是,代代相传,以及吾身。吾生为中国之人,安可不爱中国也?\by{(清)满汉蒙教科书}
\item 世界上有许多美好的地方。但是,那里有黄山么?有黄河么?有长江么?有长城么?既然这些都没有,那么,祖国就是一个不可替代的地方。\by{路遥}
\item 我每次出门旅行,总会随身携带一瓶故乡的水土,有时候在客城的旅店,就觉得那灰黑色的水土非常美丽,充满了力量。\by{林清玄}

\end{enumerate}

\subsection{人物}
\rw{钱学森}{“我很高兴能回到自己的国家,我不打算再回美国。我已被美国政府刻意地延误了我返回祖国的时间,其中的原因,我建议你们去问问你们的政府。我打算竭尽努力,去帮助中国人民建设自己的国家,使我的同胞能过上有尊严和幸福的生活。”}
\rw{闻一多}{归国之后,闻一多清醒地认识到,当时中国的社会就如同“一滩绝望的死水”,如果想追求民主独立,便要从根本上拔除帝国主义和封建军阀的统治。他就此抛弃文化救国的想法,加入中国民主同盟,成为一位民主斗士。“你看见一个倒下去,可也看得见千百个继起的!”}
\rw{关天培}{1841年年初,虎门之战彻底爆发,在英军大规模强攻之下,在声声炮火、血海厮杀中,关天培和战士们竭力奋战,最终,全部壮烈殉国。}
\rw{施一公}{已是国际著名结构生物学家、美国普林斯顿大学终身讲席教授的施一公,面对广阔的事业发展前景,面对优越的生活条件,他却作出了一个让许多人为之惊讶而敬佩的决定:放弃这一切,全职回国,回到母校清华。在他看来,“爱国是最朴素的感情,有谁不爱自己的母亲呢?”}
\rw{于谦}{“土木堡之变”发生时,他力排众议,阻止南迁,以兵部尚书身份保卫北京,破瓦剌之军,使得明王朝得以保全。}
\rw{杨靖宇}{抗日名将。
\begin{itemize}
\item 如果中国人都投降了,那还有中国吗?
\item 国家到了如此地步,除了我寻为其死,毫无其他办法。\\
\by{《抗战家书》}
\end{itemize}}

\section{文化、历史}
五千年,不是一件可以随便拍卖的古董,而是一盏会带来幸福的神灯。
\subsection{句子}
\begin{enumerate}
\item 五千年前,我们和古埃及人一样面对洪水;四千年前,我们和古巴比伦人一样玩青铜器;三千年前,我们和古希腊人一样思考哲学;两千年前,我们和罗马人一样四处征伐;一千年前,我们和阿拉伯人一样无比富足;而今天,我们和美利坚一较长短。\\
(摘自网易云音乐评论区。也许讲的没有道理,但句式很好,故摘录之。)
\item 每个国家和民族的历史传统、文化积淀、基本国情不同,其发展道路必然有着自己的特色。\by{习近平}
\item 传承中国文化的不仅仅是唐诗宋词京剧昆曲,还有与我们生活相关的每一分钟。
\item 经典之魅力在于超越时空的阻隔,永远可以与当下对话。\by{《光明日报》}
\item 总盯着过去,你会瞎掉一只眼。然而忘掉历史,你会双目失明。\by{索尔仁尼琴}
\item 中国的昨天已经写在人类的史册上,中国的今天正在亿万人民手中创造,中国的明天必将更加美好。\by{习近平}
\item 不是所有东西都可以拿来娱乐消费。一个民族如果记不得曾经的痛,就唤不起今天的梦。
\item 中华民族看似软弱,有时还带点懦弱怕事的味道。但是几经外辱,却能过显示出坚强无比的无可征服性。\by{曾仕强}
\item 传统不是可以逐气温而脱的外衣,甚至都不是因发育而定期蜕除的角质表皮。它无法随手扔掉,难以彻底决裂,除非谁打算自戕或自焚。\by{庞朴}
\end{enumerate}
\subsection{事例、人物}
\rw{邝健廉}{他作为粤剧艺术的一代宗师,在传统旦角的唱腔基础上,吸收昆曲、京剧、西洋歌剧、话剧等中的技巧,开创了中国粤剧史上花旦行当中影响最大的唱腔流派之一——“红派艺术”。}
\rw{蒯翔}{永乐年间的故宫设计师。他在建筑学上的造诣达到了炉火纯青的地步。施工前精确计算、竣工后分毫不差,巧妙使用榫卯结构、巧妙运用江南的建筑艺术,在历史的机遇面前,他创造了一个建筑史上的奇迹。}
\rw{梅兰芳}{他积极在传统京剧戏法的基础之上创新,开创了京剧的新派系“梅派”。}
\rw{张艺谋}{作为2008年北京奥运会开幕式导演,他推掉了高昂报酬和商业合作,带领团队夜以继日地工作,经过一千多个日夜的努力,开幕式以出人意料的方式,精彩地阐述了中华文明的演进和与世界文明的交融。}


